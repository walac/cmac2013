\documentclass[a4,11pt]{article}
\usepackage[utf8]{inputenc}
\usepackage[brazil]{babel}
\usepackage{epsfig}
\usepackage[top=3cm,bottom=2cm,left=3cm,right=2cm]{geometry}
\usepackage{indentfirst}
\usepackage{longtable}
\usepackage{lscape}
\usepackage{vaucanson-g}
\usepackage{amsmath}
\usepackage{amsfonts}
\usepackage{algorithmic}
\usepackage{algorithm}
\usepackage{float}
\usepackage{chngcntr}
\usepackage{lscape}
\usepackage{multicol}
%\usepackage{algcompatible}

\pagestyle{empty}

\headheight 20mm      %
\oddsidemargin 2.0mm  %
\evensidemargin 2.0mm %
\topmargin -40mm      %
\textheight 250mm     %
\textwidth 160mm      %

\floatname{algorithm}{Algoritmo}

\newenvironment{definition}[1][Definition]{\begin{trivlist}
\item[\hskip \labelsep {\bfseries #1}]}{\end{trivlist}}

\counterwithin{equation}{section}
\counterwithin{figure}{section}
\counterwithin{table}{section}

\begin{document}

\title{ \Large{\bf Uma notação matricial para representação de grafos de processo de autômatos celulares elementares}}

\author{ {\bf {\large \underline{Wander Lairson Costa}$^2$}} \hspace*{1cm} {\bf {\large
 Pedro P.B. de Oliveira$^{1,2}$}}  \\
 {\small Faculdade Computação e Informática$^1$ \& Pós-Graduação em Engenharia Elétrica$^2$} \\
 {\small Universidade Presbiteriana Mackenzie} \\
 {\small Rua da Consolação 896, Consolação - 013020-907 São Paulo, SP} \\
 {\small wander.lairson@gmail.com \hspace*{.2cm}  pedrob@mackenzie.br} }

\date{}

\maketitle

\thispagestyle{empty}
\noindent{\bf Palavras-chave:} {\it Autômatos celulares, Matrizes de adjacência, Grafos de processo}

\vspace{5ex}

\noindent{\bf Resumo:}
\textit{Autômatos celulares (ACs) são sistemas totalmente discretos que agem
localmente de forma simples e determinística, mas cujo comportamento
global resultante pode ser extremamente complexo. O conjunto de possíveis
configurações globais em um passo de tempo $t$ finito para um AC pode ser
descrito por um grafo de processo, que é um tipo de autômato finito em que
todos os estados são iniciais e finais. Propõe-se aqui uma nova representação
de grafos de processo para ACs por meio de matrizes. Esta notação
permite derivar algoritmos de formação de grafos de processo para qualquer passo
de tempo de ACs que possuem crescimento estruturado.
Conclui-se que essa nova notação facilita a derivação de algoritmos de formação de
grafos de processo e acelera o processo de experimentos computacionais com ACs.}

\section{Introdução} 

Autômatos celulares (ACs) são sistemas discretos que apresentam um comportamento
local extremamente simples, mas cujo comportamento global pode ser tão
complexo quanto uma máquina de Turing \cite{wolfram1984a}. Ele
é caracterizado por um reticulado de células cujos estados posteriores
dependem dos estados anteriores de si próprias e de suas vizinhas.
Este tipo de comportamento, que é localmente simples mas globalmente
complexo dá origem à chamada \textit{computação emergente}, e é
a base da utilização dos ACs na modelagem de diversos sistemas naturais.

A vizinhança das células de um AC por ser definida por
um raio de vizinhança que tem seu centro na célula em questão, e engloba
todas as células que está dentro do raio. Um AC
elementar possui raio igual a 1 e conjunto de estados $\{0,1\}$,
o que permite a definição de um espaço de 256 regras possíveis, que
são usualmente identificadas por um número entre 0 e 255
\cite{wolfram1984}.

O comportamento limite de ACs pode ser descrito por meio
de linguagens formais, a partir das cadeias ou
palavras formadas por seu alfabeto \cite{lewis2008}.
A classe mais simples de linguagens formais é
a das linguagens regulares, que pode ser gerada por uma gramática regular ou
pelas chamadas expressões regulares. Qualquer linguagem regular pode ser
reconhecida por um autômato finito (AF), que é uma modalidade de autômato que
possui memória restrita. As linguagens regulares são o tipo mais simples
de linguagem na hierarquia de Chomsky.

O conjunto de todas as configurações globais de um AC
elementar em um dado momento no tempo discreto pode ser descrito por
uma linguagem regular e, consequentemente, um AF
\cite{wolfram1984}.

Um AF que representa o conjunto de configurações de um passo de tempo de um
AC pode também ser representado por um grafo de processo \cite{wolfram1984}.
Um grafo de processo é um tipo de autômato onde todos os estados são
iniciais e finais \cite{hanson1992}.

Para se determinar o grafo de processo do passo $t+1$ de um AC, é
necessário partir-se do grafo de processo do passo $t$. Dentro do espaço
elementar de regras, existem ACs cujos sucessivos grafos de processo que
representam sua evolução temporal apresentam crescimento
estruturado, e é possível determinar o grafo de processo de tempo $t$ de
forma direta, sem a necessidade de gerar-se todos os $t-1$ grafos de
processo \cite{trafaniuc2004}. Este trabalho apresenta uma proposta de
representação de grafos de processo utilizando uma nova notação matricial
que permite desenvolver algoritmos que gerem os grafos de processo para
ACs elementares de forma direta, sem depender do grafo de processo do
instante anterior.

A Seção \ref{sec:ca} descreve de forma sumária os padrões de crescimento
de ACs, a Seção \ref{sec:matrix} descreve a notação de matrizes de adjacência
desenvolvida e a Seção \ref{sec:conclusion} conclui o texto.

\section{Padrões de crescimento de ACs elementares}\label{sec:ca}

O espaço de configurações iniciais de um AC
elementar pode ser qualquer combinação de sequências de 0s e 1s
representando os estados iniciais das células. O grafo de processo
que representa tal sequência pode ser visualizado na Figura
\ref{fig:initconfigautomaton}.

Após 1 passo de tempo, o conjunto de configurações que podem aparecer
torna-se mais restrito, e esse conjunto é representado por um novo grafo
de processo. A Figura \ref{fig:r184step1} mostra o grafo de processo que
representa 1 passo de tempo da regra elementar 184.

\begin{figure}[htp]
\begin{minipage}[b]{0.5\linewidth}
\centering
\begin{VCPicture}{(0,0)(2,2)}
\State[A]{(1,1)}{A}
\LoopN{A}{0,1}
\end{VCPicture}
\caption{Grafo de processo representando as possíveis configurações
iniciais.}
\label{fig:initconfigautomaton}
\end{minipage}
\hspace{0.5cm}
\begin{minipage}[b]{0.5\linewidth}
\centering
\begin{VCPicture}{(-3,-3)(3,3)}
\State[B]{(-3,0)}{B} \State[C]{(0,-2)}{C}
\State[A]{(0,2)}{A} \State[D]{(3,0)}{D}
\ArcL{A}{C}{1}
\LoopW{B}{1} \ArcL{B}{A}{0}
\ArcL{C}{B}{1} \ArcL{C}{D}{0}
\LoopE{D}{0} \ArcL{D}{C}{1}
\end{VCPicture}
\caption{Grafo de processo representando as possíveis configurações
após 1 passo de tempo para a regra 184.}
\label{fig:r184step1}
\end{minipage}
\end{figure}

Os grafos de processo de determinadas regras do espaço elementar possuem como
característica um padrão de crescimento. Trafaniuc \cite{trafaniuc2004}
identificou 26 regras com essa característica, e conseguiu obter algoritmos
para geração do grafo de processo de tempo $t$ para 12 delas, por meio da
inspeção visual dos grafos de processo. A seguir é apresentada uma notação
de representação matricial de grafos de processo que visa facilitar essa
análise manual.

\section{Grafos de processo por meio de matrizes de adjacência}\label{sec:matrix}

Grafos podem ser representados por uma matriz $N \times N$, onde $N$
é o número de estados do grafo. Cada posição $A_{ij}$ da matriz recebe o valor
1 quando existe uma transição do estado $i$ para o estado $j$, ou 0 caso o
contrário. Este tipo de notação, porém, é insuficiente para representar grafos
de processo, dado que: (1) Em grafos de processo, é possível haver mais de
uma transição por par de estados. (2) Nos índices da matriz não está
codificado o valor da transição, que é o valor esperado na fita de
entrada de um AF.

Para contornar este problema, foi desenvolvida uma nova notação de matrizes de adjacência,
chamadas de \textit{matrizes de adjacência de evolução temporal}.

Uma matriz de adjacência de evolução temporal é uma notação de matriz de
adjacência para representar grafos de processo que possui uma relação
de isomorfismo entre a matriz e sua representação em grafo.
Formalmente, dado o conjunto de matrizes de adjacência
de evolução temporal $\mathbb{M}$ e o conjunto de grafos de processo $\mathbb{G}$,
existe uma transformação $T: \mathbb{G} \mapsto \mathbb{M}$ e uma
transformação $T^{-1}: \mathbb{M} \mapsto \mathbb{G}$. 

O elemento chave para conseguir essa relação de isomorfismo
é encontrar uma representação do conjunto de transições e
seus respectivos valores em um número real que possa ser codificado
nos índices da matriz, e que possa ser decodificado quando for
aplicada a transformação $T^{-1}$. Formalmente, dado o conjunto de transições
$\mathbb{L}$ do estado $i$ para o estado $j$, $\forall i,j$, encontrar uma
relação $\eta:\mathbb{L} \mapsto \Re$ que seja bijetora.

A resposta para esta questão está em criar um mapeamento dos valores
possíveis de transições para o conjunto de números naturais
$\mathbb{L} \mapsto \mathbb{N}$ e então codificar os valores das transições
em um número binário. Cada bit do número binário corresponde a um valor de
transição. Se o valor no $k$-ésimo bit for 1, existe uma
transição (mapeada no conjunto dos naturais) de valor $k$ do estado de
origem para o estado destino em questão. A operação $\eta$ de codificação dos
arcos $\mathbb{L}$ em um número natural é o polinômio de base 2
$\eta(\mathbb{L}_{ij}) = \sum_{\forall k \in (\mathbb{L}_{ij} \mapsto \mathbb{N})} 2^k$,
onde $\mathbb{L}_{ij}$ é o conjunto de transições do estado $i$ para o estado $j$,
o que permite a definição de um espaço de 256 regras possíveis, que são usualmente
identificadas por um número entre 0 e 255 \cite{wolfram1984}.

Nota-se que como $k \in \mathbb{N}$, então $\eta(\mathbb{L}_{ij}) \in \mathbb{Z}$,
ou seja, $\eta(\mathbb{L}_{ij})$ está restrita ao domínio dos inteiros. Se
$k \in \mathbb{Z}$ ($\mathbb{L}_{ij} \mapsto \mathbb{Z}$), então
$\eta(\mathbb{L}_{ij}) \in \Re$, pois admitiria-se a possibilidade de expoentes negativos.
Como computacionalmente é mais custoso trabalhar com números reais do que com inteiros
e não há benefício aparente em expandir $\eta$ para o domínio dos reais,
já que $\mathbb{N}$ é contavelmente infinito \cite{lewis2008}, optou-se
por restringir $\eta$ ao domínio dos números inteiros. Outra observação
é a de que a base numérica não necessariamente deve ser base 2, poderia-se trabalhar com outras
bases. Todavia, a única implicação na mudança de base é a quantidade de dígitos
necessária para representar um dado conjunto $2^{\mathbb{L}}$ de possíveis combinações de
transições e, do ponto de vista de arquitetura computacional, a base 2 é a
base mais natural para se trabalhar.

\begin{algorithm}
\caption{Algoritmo para gerar a matriz de adjacência de evolução temporal a partir
de um grafo de processo.}
\label{alg:stom}
\begin{algorithmic}
\STATE $n \leftarrow \mbox{Length}(\mathbb{Q})$ \COMMENT{número de estados}
\STATE $A \leftarrow \mbox{ Matriz nula } n \times n$
\FORALL{$i,j \in \mathbb{Q}$}
\IF{$\mathbb{L}_{ij} \neq \emptyset$}
\STATE $A_{ij} = \eta(\mathbb{\mathbb{L}}_{ij})$
\ENDIF
\ENDFOR
\end{algorithmic}
\end{algorithm}

O Algoritmo \ref{alg:stom} refere-se à transformação de um grafo
de processo em uma matriz de evolução temporal. Já o Algoritmo \ref{alg:mtos}
representa o processo inverso, ou seja, transforma uma
matriz de evolução temporal em um grafo de processo; este último
faz uso de uma operação chamada \emph{MakeTransition(i,j,n)}, que cria uma transição
do estado $i$ para o estado $j$ com valor $n$.

De posse dos algoritmos de transformação entre matrizes de adjacências e grafos
de processo, foi possível obter o algoritmo que gere as matrizes de adjacências de tempo
T para os 26 ACs identificados em \cite{trafaniuc2004}. Como
ilustração, o Algoritmo \ref{alg:r184} representa o processo para a regra 184.

\begin{figure*}[ttt!]

\begin{minipage}[t]{3.00in}
\begin{algorithm}[H]
\caption{Algoritmo para gerar o grafo de processo a partir de uma matriz de
adjacência de evolução temporal.}
\label{alg:mtos}
\begin{algorithmic}
\FOR{$i=1$ to $n$}
    \FOR{$j=1$ to $n$}
        \IF{$A_{ij} \neq 0$}
            \REPEAT
                \STATE $q = \lfloor A_{ij} \div 2 \rfloor$
                \STATE $r = Mod(A_{ij},2)$
                \IF{$r \neq 0$}
                    \STATE $n=0$
                    \WHILE{$2^n \le A_{ij}$}
                        \STATE $n=n+1$
                    \ENDWHILE
                    \STATE MakeTransition$(i,j,n-1)$
                \ENDIF
            \UNTIL{$q \neq 0$}
        \ENDIF
    \ENDFOR
\ENDFOR
\end{algorithmic}
\end{algorithm}
\end{minipage}
\hfill
\begin{minipage}[t]{3.00in}
\begin{algorithm}[H]
\caption{Algoritmo para gerar a matriz de adjacência de evolução temporal do
grafo de processo de tempo $t$ para a regra 184.}
\label{alg:r184}
\begin{algorithmic}
\STATE $A \leftarrow \mbox{ matriz nula } n \times n \mbox{, onde } n=2(t+1)$
\STATE $i \leftarrow t+1$
\FOR{$j=1$ to t}
    \STATE $A_{ij} \leftarrow 1$
    \STATE $i \leftarrow i+1$
\ENDFOR
\FOR{$i=t+1$ to $n$}
    \STATE $A_{i,t+1} \leftarrow 2$
\ENDFOR
\STATE $j \leftarrow t+2$
\FOR{$i=1$ to $t$}
    \STATE $A_{ij} \leftarrow 2$
    \STATE $j \leftarrow j+1$
\ENDFOR
\STATE $A_{n,n} \leftarrow 1$
\STATE $A_{n-1,n} \leftarrow 1$
\STATE $A_{n,n-1} \leftarrow 2$
\end{algorithmic}
\end{algorithm}
\end{minipage}
\hfill
\end{figure*}

\section{Conclusão}\label{sec:conclusion}

A presente proposta de notação de grafos de processo permitiu o
desenvolvimento dos algoritmos de formação de grafos de processo de tempo
T para as 26 regras descritas em \cite{trafaniuc2004}, onde até então
somente tinha sido possível encontrar os algoritmos para 12 dessas regras,
o que mostra uma evolução em relação à técnica desenvolvida anteriormente.
Este fato evidencia a maior facilidade de análise da notação matricial em
relação à notação em grafo. Para a validação do algoritmo, foram comparadas
as matrizes de adjacência geradas utilizando esta técnica com as matrizes
geradas pelo método convencional de construção de grafos de processo para
os primeiros 100 passos de tempo para cada regra do grupo de 26 estudadas.

De posse do algoritmo de geração da matriz de passo T para uma
determinada regra e do algoritmo de transformação de uma matriz de
adjacência de evolução temporal em um grafo de processo, é possível
gerar o grafo de processo para qualquer instante de tempo com complexidade
$O(1)$ em relação ao passo de tempo, em contraste com a complexidade
$O(n)$ da abordagem tradicional.

Com isso, recursos computacionais em experimentos envolvendo ACs
elementares podem ser drasticamente reduzidos, e estudos,
como por exemplo, sobre comportamento limite de ACs por meio de linguagens
regulares tornam-se mais fáceis.

%\def\refname{Referências bibliográficas}
\bibliographystyle{plainpor}
\bibliography{ca_stepT}
%\addcontentsline{toc}{section}{Referências bibliográficas} 

\end{document}
